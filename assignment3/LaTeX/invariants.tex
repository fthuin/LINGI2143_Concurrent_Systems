\section{T- and P-invariants}
\label{sec:T- and P-invariants}

\subsection{P-Semi-Flows}

\begin{lstlisting}
{coroners free}
{clerks free} crime incident
{inspectors free} report2
intervention*2 intervention2*2 {officers free} {patrols free}*2 report2
intervention intervention2 {patrols free} {vehicle free}
\end{lstlisting}

These invariants show how the Petri net can get back the global same amount of tokens. It is the result of a set of linear \enquote{flow} equations from the incidence matrix. \newline

For example, \verb#{coroners free}# place always have the same amount of tokens (because the arrow leaves this place to a transition that has an arrow to this place). \newline

Another example is that the amount of tokens in \verb#{clerks free}# and \verb#crime# and \verb#incident# is always the same. It can be explained by the fact that the arrows from \verb#{clerks free}# leaves weither to \verb#crime# or to \verb#incident#, which next transitions lead to send the token back to \verb#{clerks free}#. \newline

\subsection{T-Semi-Flows}

\begin{lstlisting}
{incident call} {patrol found} report
{crime call} {form patrol} investigation {patrol found2} {write report}
{form patrol} {free patrol}
\end{lstlisting}

These invariants show how the Petri can get back to the exact same state. \newline

Activating a transition \verb#{incident call}#, then a \verb#{patrol found}# transition, then a \verb#report# transition will take the Petri net in the exact same state it was before.
